\documentclass[letterpaper,12pt,fleqn]{article}
\usepackage{amsmath}
\usepackage{float}
\usepackage{graphicx}
\usepackage{subcaption}
\usepackage{booktabs}
\usepackage{xcolor}
\usepackage{framed}
\usepackage{gfsartemisia-euler}
\usepackage[T1]{fontenc}
\usepackage[shortlabels]{enumitem}

\setlength{\parskip}{1em}

\definecolor{shadecolor}{gray}{.9}
\newcommand{\mybox}[1]{\par\noindent\colorbox{shadecolor}
{\parbox{\dimexpr\textwidth-2\fboxsep\relax}{\tt #1}}}
\newcommand{\code}[1]{\colorbox{shadecolor}{\tt #1}}

\newcommand\f{\frac}

\begin{document}

%%%%%%%%%%%%%%%%%%%%%%% Title %%%%%%%%%%%%%%%%%%%%%%%%%%%%%%%%%%%
								%
\centerline{\bf \huge MATH 5340 --- Comp Method II}		%
\bigskip							%
\centerline{\Large Homework \#2} 				%
\medskip							%
\centerline{\textsc{Xiukun Hu, Mallory Lai, Geeta Monpara}}	%
\bigskip							%
								%
%%%%%%%%%%%%%%%%%%%%%%%%%%%%%%%%%%%%%%%%%%%%%%%%%%%%%%%%%%%%%%%%%

\section{Step 1: 1D}
\subsection{Instructions}
Basic command to run 1D code is:
\mybox{make 1D}

Or you can compile it first by \code{make}, and then run the executable \code{./theta1D}\;.

Some parameters can be changed using Makefile macros:
\mybox{make -B LENGTH=3.0 HEIGHT=0.5 1D}

Adjustable parameters during compile time include:
\begin{table}[h]
  \begin{tabular}{p{2cm} p{3.3cm} p{7cm}}
    \toprule
    \textbf{Name} & \textbf{Example} & \textbf{Description}\\
    \midrule
    Data    &	\code{DATA=XIUKUN}    &	  \textit{Decide whose initial and boundary conditions to be used.
      Options include:
  \texttt{XIUKUN, GEETA} and \texttt{MALLORY}}.\\
    Length  &	\code{LENGTH=3.0}     &	  \textit{Domain length in inch.}\\
    Time    &	\code{TIME=70}	      &	  \textit{Total time computed, in minute.}\\
    Alpha   &	\code{ALPHA=.2}	      &	  \textit{Thermal Diffusivity in inch/min.}\\
    \bottomrule
  \end{tabular}
\end{table}

$dx$ (inch), $dt$ (minute) and $\theta$ are required as input in runtime.
If input is illegal, you need to input again.
If $\theta$ is \code{-1}, then the special case
$\theta=\f12 -\frac{(\Delta x)^2}{12\Delta t}$
will be applied.

\subsection{Result}
The initial conditions are acquired using quadratic interpolation
with central and exterior temperature data,
assuming that both end have the same temperature.
The boundary conditions are acquired using exponential regression with boundary data collected.


The output includes two files: \texttt{theta1D.txt} and \texttt{theta1D.bin}.
The text file contains information of the model, which is used in visualizing the data.
The binary file contains output data in binary format.
Please see Section 3 about how to visualize resulting data.

Figure~\ref{fig:data-comp-1d} illustrates the comparison between numerical central temperatures and central temperatures collected from homework 1.
The settings for numerical solution are $dx=.1, dt=.01, \theta=-1$, and the thermal diffusivity $\alpha=.2$.

\begin{figure}[H]
  \begin{subfigure}[b]{\textwidth}
    \centering
    \includegraphics[width=0.8\linewidth]{xiukun1D}
    \subcaption{Xiukun's Data}
  \end{subfigure}
  \caption{Data Comparison}
\end{figure}
\begin{figure}[H]\ContinuedFloat
  \begin{subfigure}[b]{\textwidth}
    \centering
    \includegraphics[width=0.8\linewidth]{geeta1D}
    \caption{Geeta's Data}
  \end{subfigure}

  \begin{subfigure}[b]{\textwidth}
    \centering
    \includegraphics[width=0.8\linewidth]{mallory1D}
    \caption{Mallory's Data}
  \end{subfigure}
  \caption{Data Comparison (cont.)}
  \label{fig:data-comp-1d}
\end{figure}
There is a video file named \texttt{theta1D.mp4} showing the cooling process in 1D with Mallory's initial condition and boundary condition.


\subsection{Performance}
The following elapsed times are recorded using GNU \textit{time}. 
The elapsed real time in second is presented.
$\theta=1$ is applied. And $nt=70000$ for all tests.

\begin{table}[H]
  \centering
  \begin{tabular}{l c c c c c c}
    \toprule
    \textbf{nx} & 300 & 600 & 1200 & 2400 & 4800 & 9600\\
    \midrule
    \textbf{real} & 0.80 & 1.43 & 2.68 & 5.32 & 10.57 & 21.19\\
    \bottomrule
  \end{tabular}
  \caption{Elapsed time of 1D code}
  \label{tab:time-1d}
\end{table}


\section{Step 2: 2D}
\subsection{Instructions}
Basic command to run 1D code is:
\mybox{make 2D}

Or you can compile it first by \code{make}, and then run the executable \code{./theta2D}\;.

Some parameters can be changed using Makefile macros:
\mybox{make -B LENGTH=3.0 HEIGHT=0.5 2D}

Adjustable parameters during compile time include:
\begin{table}[h]
  \begin{tabular}{p{2cm} p{3.3cm} p{7cm}}
    \toprule
    \textbf{Name} & \textbf{Example} & \textbf{Description}\\
    \midrule
    Data    &	\code{DATA=XIUKUN}    &	  \textit{Decide whose initial and boundary conditions to be used.
      Options include:
  \texttt{XIUKUN, GEETA} and \texttt{MALLORY}}.\\
    Length  &	\code{LENGTH=3.0}     &	  \textit{Domain length in inch.}\\
    Height  &	\code{HEIGHT=.5}      &	  \textit{Domain height in inch.}\\
    Time    &	\code{TIME=70}	      &	  \textit{Total time computed, in minute.}\\
    Alpha   &	\code{ALPHA=.2}	      &	  \textit{Thermal Diffusivity in inch/min.}\\
    \bottomrule
  \end{tabular}
\end{table}

$dy$ (inch), $dx$ (inch), $dt$ (minute) and $\theta$ are required as input in runtime.
If input is illegal, you need to input again.

\subsection{Result}
The boundary conditions are the same as in 1D (exponential interpolation).
The initial conditions however were acquired using the assumption that
there is a time point when the temperature is the same everywhere in the muffin.
This time point was calculated by exponentially interpolating the center temperature
and finding the time when the center temperature equals the exterior temperature.
That time was set to be $t_0$ and the initial condition is just constant.

The output includes two files: \texttt{theta2D.txt} and \texttt{theta2D.bin}.
The text file contains information of the model, which is used in visualizing the data.
The binary file contains output data in binary format.
Please see Section 3 about how to visualize resulting data.

Figure~\ref{fig:data-comp-2d} shows the comparison between numerical central temperatures and central temperatures collected from homework 1.
The settings for numerical solution are $dx=.01, dy=.01, dt=.01, \theta=1/2$, and the thermal diffusivity $\alpha=.007$.

\begin{figure}[H]
  \begin{subfigure}[b]{\textwidth}
    \centering
    \includegraphics[width=0.8\linewidth]{xiukun2D}
    \subcaption{Xiukun's Data}
  \end{subfigure}
  \begin{subfigure}[b]{\textwidth}
    \centering
    \includegraphics[width=0.8\linewidth]{mallory2D}
    \caption{Mallory's Data}
  \end{subfigure}
  \caption{Data Comparison}
\end{figure}

\begin{figure}[H]\ContinuedFloat
  \begin{subfigure}[b]{\textwidth}
    \centering
    \includegraphics[width=0.8\linewidth]{geeta2D}
    \caption{Geeta's Data}
  \end{subfigure}
  \caption{Data Comparison (cont.)}
  \label{fig:data-comp-2d}
\end{figure}

There is a video file named \texttt{theta2D.mp4} showing the first of the cooling process in 2D with Geeta's initial condition and boundary condition.
Only 30 minutes are recorded in this video since otherwise the file will be too large.

\subsection{Performance}
The following elapsed times are recorded using GNU \textit{time}. 
The elapsed real time in second is presented.
$\theta=1$ is applied. And $nt=8000$ for all tests.

\begin{table}[H]
  \centering
  \begin{tabular}{l c c c c }
    \toprule
    \textbf{$nx\times ny$} & $150\times 25$ & $300\times 50$ & $600\times 100$ & $1200\times 200$\\
    \midrule
    \textbf{real} & 0.82 & 2.61 & 10.13 & 48.04\\
    \bottomrule
  \end{tabular}
  \caption{Elapsed time of 2D code}
  \label{tab:time-2d}
\end{table}

\section*{Visualization}
In order to visualize the output, use the Matlab code \verb|Heat_Visualize.m|.
There are two parameters for this Matlab code:\\
\indent\code{nD}: default 1, legal values are 1 or 2.
This parameter specifies which data (1D or 2D) you want to visualize.\\
\indent\code{timestep}: default 0.1, legal value is any positive number.
This parameter determines the time step for the movie.
\end{document}

